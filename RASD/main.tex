\documentclass[11pt]{article}


% -------------------------------------------------------------------------
% Load packages that util files or commandsFile need
% -------------------------------------------------------------------------
\usepackage[T1]{fontenc}
\usepackage[utf8]{inputenc}
\usepackage{xspace}      % for abbrev.tex
\usepackage{soul}        % for highlight.tex
\usepackage{tikz}        % for highlight.tex (if using TikZ diagrams)
\usepackage{xcolor}      % safer color handling
\usepackage{enumitem}
\usepackage{fancyhdr}
\setlength{\headheight}{13.59999pt}
\addtolength{\topmargin}{-1.59999pt}
\usepackage{tabularx}
% ... (any other packages you normally need) ...
% -------------------------------------------------------------------------
% Bring in helper macro files (before \begin{document})
% -------------------------------------------------------------------------
\input{util/abbrev}
\input{util/comments}
\input{util/highlight}
% Load hyperref ONCE, near the end of the preamble
\usepackage[colorlinks=true, linkcolor=blue, citecolor=blue, urlcolor=blue]{hyperref}
% -------------------------------------------------------------------------
% Bring in the main formatting commands (which sets geometry, headers, etc.)
% -------------------------------------------------------------------------
\input{commandsFile}

\begin{document}

%TITLE PAGE

\begin{titlepage}


%LOGO

\begin{table}[t!]
\centering
\begin{tabular}{>{\raggedleft\arraybackslash}p{0.55\textwidth}|>{\raggedright\arraybackslash}p{0.45\textwidth}}
\hline
\textcolor{Blue}{\textbf{\small{Best Bike Paths + project by Shuvro Ahmed, Umme Humaira Himi, Ahmed Hassan}}} & \includegraphics[width=0.5\textwidth]{Images/PolimiLogo} \\
\hline
\end{tabular}
\end{table}
\vspace{7cm}
%TITLE 

\begin{flushleft}

%Replace the text string with your title
{\textcolor{Blue}{\textbf{\Huge{Requirement Analysis and Specification
        Document}}}} \\ [1cm]

\end{flushleft}

\end{titlepage}

%Define deliverable specific info
%Replace cell contents where needed
\begin{table}[h!]
\begin{tabular}{|p{0.3\textwidth}|p{0.65\textwidth}|}
\hline
    \textbf{Deliverable:} & RASD \\
    \textbf{Title:} & Requirement Analysis and Verification Document \\
    \textbf{Authors:} & Shuvro Ahmed, Umme Humaira Himi, Ahmed Hassan \\
    \textbf{Version:} & 1.0 \\
    \textbf{Date:} & 23-December-2025 \\
    \textbf{Download page:} & https://github.com/H-I-M-I/HassanAhmedHimi \\
    \textbf{Copyright:} & Copyright © 2025, Shuvro Ahmed, Umme Humaira Himi, Ahmed Hassan – All rights reserved \\
\hline
\end{tabular}
\end{table}




\setcounter{page}{2}


%------------------------------------------------------------------------------------------------------------------------------------------------
\newpage
\addcontentsline{toc}{section}{Table of Contents}
\tableofcontents
\newpage
\addcontentsline{toc}{section}{List of Figures}
\listoffigures
\addcontentsline{toc}{section}{List of Tables}
\listoftables

%------------------------------------------------------------------------------------------------------------------------------------------------
\clearpage
{\color{Blue}{\section{Introduction}}}
\label{sect:introduction}
\input{Files/introduction}
{\subsection{Purpose}}
Cycling is an increasingly popular mode of urban transport and recreation, yet cyclists often struggle to find routes that are both safe and comfortable. Existing information about bike paths is scattered and unreliable: some paths are well maintained while others contain potholes, uneven surfaces, or suffer frequent vehicle interference. There is no centralized platform that continuously collects and validates this information in a structured way.

The Best Bike Paths (BBP) system addresses this gap by providing a unified, data-driven platform where cyclists can record trips, assess path quality, and share observations with the community. BBP supports two data-collection modes: manual entry, where users input details about paths they travel, and automatic acquisition, where the system records GPS and sensor data during rides to detect irregular surfaces. It also integrates external weather information to enrich trip records and improve analysis accuracy.

By aggregating reports from multiple users and resolving inconsistencies using data freshness and confirmation counts, BBP maintains an up-to-date, reliable inventory of cycling paths to help cyclists plan routes that are both efficient and safer.

{\subsection{Goal}}
G1: Enable cyclists to travel along paths that are safe, smooth, and well maintained.

G2: Inform cyclists about current path conditions and route quality before they start a trip.

G3: Keep path information accurate and consistent over time by combining reports from multiple users.

G4: Provide personalized route suggestions that balance safety and effectiveness (distance, effort, weather).


%------------------------------------------------------------------------------------------------------------------------------------------------
\clearpage
{\color{Blue}{\section{Overall Description}}}
\label{sect:overview}
\input{Files/overview}

%------------------------------------------------------------------------------------------------------------------------------------------------
\clearpage
{\color{Blue}{\section{Specific Requirements}}}
\label{sect:requirements}
\input{Files/requirements}

%------------------------------------------------------------------------------------------------------------------------------------------------
\clearpage
{\color{Blue}{\section{Formal Analysis Using Alloy}}}
\label{sect:alloy}
\input{Files/alloy}

%------------------------------------------------------------------------------------------------------------------------------------------------
\clearpage
{\color{Blue}{\section{Effort Spent}}}
\label{sect:effort}
\input{Files/effort}


%------------------------------------------------------------------------------------------------------------------------------------------------
\clearpage
\addcontentsline{toc}{section}{References}
%\bibliographystyle{plain}
%\bibliography{main}
%-----------------------------------------------



\end{document}
